% essentially from https://en.wikibooks.org/wiki/LaTeX/Presentations
\documentclass[10pt]{beamer}

\usetheme{Goettingen}
\usecolortheme{wolverine}

% Add extra packages here
%\usepackage{pgfpages}
\usepackage{hyperref}
%% configure display of speaker notes:
%% see https://brandonrozek.com/blog/notes-beamer-latex/
\setbeameroption{hide notes} % Only slides
%\setbeameroption{show only notes} % Only notes
%\setbeameroption{show notes on second screen=right} % Both


\title{Presentation on  topic XXX}
\author{Michael Adam (\href{mailto:obnox@samba.org}{obnox@samba.org})}
\begin{document}
    \setbeamercovered{transparent}
    \maketitle

    \note{Welcome}
    \note{example beamer slides}


    \section{slides (frames) and lists}


    \begin{frame}
	    \frametitle{This is the first slide}
		Here you can put any text/equation etc. 

        writing slides in \LaTeX.

        some formula: $a^2 + b^2 = c^2$.

        \note[item]{example LaTeX beamer slides.}
        \note[item]{beautiful output, plain text source, SCM}

    \end{frame}
	\begin{frame}
		\frametitle{This is the second slide}
		\framesubtitle{A bit more information about this}
		Some random text.

        more formula: $\sum_{i=1}^{n}i = \frac{n(n+1)}{2}$.

    \end{frame}
    \begin{frame}
        \frametitle{another slide}
        \framesubtitle{demo of uncovering list items}
        \begin{itemize}
          \item<2-> an item.
          \item<3-> another item.
          \item<4-> guess what.
        \end{itemize}
    \end{frame}
\end{document}

\section{more features}

\begin{frame}
\frametitle{blocks/text boxes}
\begin{block}{some text block in a box}

This is the text in the box

It can be any text and \LaTeX\ code.

\end{block}
\end{frame}